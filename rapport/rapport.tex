\documentclass{article}

\usepackage[utf8x]{inputenc}
\usepackage[T1]{fontenc}
\usepackage[francais]{babel}
\usepackage{xcolor}
\usepackage{listings}
\usepackage{mathptmx}
\usepackage{anyfontsize}
\usepackage{t1enc}
\usepackage[top=2cm, bottom=2cm, left=2cm, right=2cm]{geometry}
\usepackage{titlesec}
\usepackage{titling}
\usepackage{graphicx}

\newcommand{\changeurlcolor}[1]{\hypersetup{urlcolor=#1}}

\renewcommand\maketitlehooka{\null\mbox{}\vfill}
\renewcommand\maketitlehookd{\vfill\null}

\definecolor{codegreen}{rgb}{0,0.6,0}
\definecolor{codegray}{rgb}{0.5,0.5,0.5}
\definecolor{codepurple}{rgb}{0.58,0,0.82}
\definecolor{backcolour}{rgb}{0.95,0.95,0.92}
\definecolor{codekeywords}{rgb}{0.1,0.53,0.92}

\lstdefinestyle{c++}{
    backgroundcolor=\color{backcolour},   
    commentstyle=\color{codegreen},
    keywordstyle=\color{codekeywords},
    numberstyle=\tiny\color{codegray},
    stringstyle=\color{codepurple},
    basicstyle=\ttfamily\footnotesize,
    breakatwhitespace=false,         
    breaklines=true,                 
    captionpos=b,                    
    keepspaces=true,                 
    numbers=left,                    
    numbersep=5pt,                  
    showspaces=false,                
    showstringspaces=false,
    showtabs=false,                  
    tabsize=2,
    texcl=false,
    inputencoding=utf8,
    extendedchars=true,
    literate=
  {á}{{\'a}}1 {é}{{\'e}}1 {í}{{\'i}}1 {ó}{{\'o}}1 {ú}{{\'u}}1
  {Á}{{\'A}}1 {É}{{\'E}}1 {Í}{{\'I}}1 {Ó}{{\'O}}1 {Ú}{{\'U}}1
  {à}{{\`a}}1 {è}{{\`e}}1 {ì}{{\`i}}1 {ò}{{\`o}}1 {ù}{{\`u}}1
  {À}{{\`A}}1 {È}{{\'E}}1 {Ì}{{\`I}}1 {Ò}{{\`O}}1 {Ù}{{\`U}}1
  {ä}{{\"a}}1 {ë}{{\"e}}1 {ï}{{\"i}}1 {ö}{{\"o}}1 {ü}{{\"u}}1
  {Ä}{{\"A}}1 {Ë}{{\"E}}1 {Ï}{{\"I}}1 {Ö}{{\"O}}1 {Ü}{{\"U}}1
  {â}{{\^a}}1 {ê}{{\^e}}1 {î}{{\^i}}1 {ô}{{\^o}}1 {û}{{\^u}}1
  {Â}{{\^A}}1 {Ê}{{\^E}}1 {Î}{{\^I}}1 {Ô}{{\^O}}1 {Û}{{\^U}}1
  {œ}{{\oe}}1 {Œ}{{\OE}}1 {æ}{{\ae}}1 {Æ}{{\AE}}1 {ß}{{\ss}}1
  {ç}{{\c c}}1 {Ç}{{\c C}}1 {ø}{{\o}}1 {å}{{\r a}}1 {Å}{{\r A}}1
  {€}{{\EUR}}1 {£}{{\pounds}}1,
}
\lstset{style=c++}


\title{Développement des Bases de Données}
\author{Arquillière Mathieu}

\begin{document}

\begin{titlepage}
  \maketitle
\end{titlepage}

\newpage

\section{TP1}

\section{TP2}

\subsection{Exercice 1}
Code :
\begin{lstlisting}[language=SQL,
    morekeywords={DECLARE, LOOP, TYPE, FOR, IF, IS, OPEN, FETCH, DBMS_OUTPUT, PUT_LINE}]
DECLARE
    nom emp.ename%TYPE;
    salaire emp.sal%TYPE;
    commission emp.comm%TYPE;
    departement dept.dname%TYPE;
BEGIN 
    SELECT ename,sal,comm,dname INTO nom,salaire,commission,departement FROM Emp NATURAL JOIN Dept
    WHERE ename='MILLER';
    DBMS_OUTPUT.PUT_LINE('Nom : ' || nom || ' Salaire : ' || salaire || ' Commission : ' || commission || 'Departement : ' || departement);
END;
/
\end{lstlisting}

Résultat :
\begin{lstlisting}[language=SQL,
    morekeywords={DECLARE, LOOP, TYPE, FOR, IF, IS, OPEN, FETCH, DBMS_OUTPUT, PUT_LINE}]
Nom : MILLER Salaire : 1300 Commission : Departement : ACCOUNTING
\end{lstlisting}

\subsection{Exercice 2}
Code :
\begin{lstlisting}[language=SQL,
    deletekeywords={char},
    morekeywords={DECLARE, LOOP, TYPE, FOR, IF, IS, OPEN, FETCH, DBMS_OUTPUT, PUT_LINE}]
DECLARE
    num1 temp.num_col1%TYPE;
    num2 temp.num_col2%TYPE;
    char temp.char_col%TYPE;
BEGIN
	FOR i IN 1..10 LOOP
		IF MOD(i, 2) = 0 THEN
			INSERT INTO temp VALUES (i, i * 100, CONCAT(TO_CHAR(i), ' est pair'));
		ELSE
			INSERT INTO temp VALUES (i, i * 100, CONCAT(TO_CHAR(i), ' est impair'));
		END IF;
	END LOOP;
	COMMIT;
END;
/
\end{lstlisting}

Résultat :
\begin{lstlisting}[language=SQL,
    morekeywords={DECLARE, LOOP, TYPE, FOR, IF, IS, OPEN, FETCH, DBMS_OUTPUT, PUT_LINE}]
SQL> select * from temp;

  NUM_COL1   NUM_COL2 CHAR_COL
---------- ---------- -------------------------------------------------------
		 5000 KING
	 1	  100 1 est impair
	 2	  200 2 est pair
	 3	  300 3 est impair
	 4	  400 4 est pair
	 5	  500 5 est impair
	 6	  600 6 est pair
	 7	  700 7 est impair
	 8	  800 8 est pair
	 9	  900 9 est impair
	10	 1000 10 est pair
\end{lstlisting}

\subsection{Exercice 3}
Code :
\begin{lstlisting}[language=SQL,
    deletekeywords={char},
    morekeywords={DECLARE, LOOP, TYPE, FOR, IF, IS, OPEN, FETCH, DBMS_OUTPUT, PUT_LINE}]
DECLARE
	Cursor c IS SELECT sal, empno, ename FROM emp ORDER BY sal DESC;
	salaire emp.sal%TYPE;
	numero emp.empno%TYPE;
	nom emp.ename%TYPE;
BEGIN
	OPEN c;
	FOR i IN 1..5 LOOP
		FETCH c INTO salaire, numero, nom;
		INSERT INTO temp VALUES (salaire, numero, nom);
	END LOOP;
END;
/
\end{lstlisting}

Résultat :
\begin{lstlisting}[language=SQL,
    morekeywords={DECLARE, LOOP, TYPE, FOR, IF, IS, OPEN, FETCH, DBMS_OUTPUT, PUT_LINE}]
SQL> select * from temp;

  NUM_COL1   NUM_COL2 CHAR_COL
---------- ---------- -------------------------------------------------------
      5000	 7839 KING
      3000	 7902 FORD
      3000	 7788 SCOTT
      2975	 7566 JONES
      2850	 7698 BLAKE
\end{lstlisting}

\subsection{Exercice 4}
Code :
\begin{lstlisting}[language=SQL,
    deletekeywords={char},
    morekeywords={DECLARE, LOOP, TYPE, FOR, IF, IS, OPEN, FETCH, DBMS_OUTPUT, PUT_LINE}]
DECLARE
    Cursor c IS SELECT UNIQUE sal, NVL(comm, 0), empno, ename FROM emp WHERE sal + NVL(comm, 0) > 2000;
    salaire emp.sal%TYPE;
    numero emp.empno%TYPE;
    nom emp.ename%TYPE;
    comm emp.comm%TYPE;
BEGIN
	OPEN c;
	LOOP
		FETCH c INTO salaire, comm, numero, nom;
		EXIT WHEN (c%notfound);
		INSERT INTO temp VALUES (salaire + comm, numero, nom);
	END LOOP;
END;
/
\end{lstlisting}

Résultat :
\begin{lstlisting}[language=SQL,
    morekeywords={DECLARE, LOOP, TYPE, FOR, IF, IS, OPEN, FETCH, DBMS_OUTPUT, PUT_LINE}]
SQL> select * from temp;

  NUM_COL1   NUM_COL2 CHAR_COL
---------- ---------- -------------------------------------------------------
      2975	 7566 JONES
      2650	 7654 MARTIN
      2850	 7698 BLAKE
      2450	 7782 CLARK
      3000	 7788 SCOTT
      5000	 7839 KING
      3000	 7902 FORD
      2200	 7000 Zangla
\end{lstlisting}

\subsection{Exercice 5}
Code :
\begin{lstlisting}[language=SQL,
    deletekeywords={char},
    morekeywords={DECLARE, LOOP, TYPE, FOR, IF, IS, OPEN, FETCH, DBMS_OUTPUT, PUT_LINE}]
DECLARE
	Cursor c IS SELECT sal, ename, empno, mgr FROM emp;
	salaire emp.sal%TYPE;
	nom emp.ename%TYPE;
	empno emp.empno%TYPE;
	mgrEmp emp.mgr%TYPE;
	chaineMgr emp.mgr%TYPE;
BEGIN
	OPEN c;
	LOOP
		FETCH c INTO salaire, nom, empno, mgrEmp;
		EXIT WHEN(c%NOTFOUND);
		IF salaire >= 4000 THEN
			SELECT mgr INTO chaineMgr FROM emp WHERE empno=7902;
			LOOP
				EXIT WHEN(chaineMgr IS NULL OR chaineMgr=mgrEmp);
				SELECT mgr INTO chaineMgr FROM emp where empno=chaineMgr;
			END LOOP;
			IF chaineMgr=mgrEmp OR mgrEmp IS NULL THEN
				INSERT INTO temp VALUES (null, salaire, nom);
			END IF;
		END IF;
	END LOOP;
END;
/
\end{lstlisting}

Résultat :
\begin{lstlisting}[language=SQL,
    morekeywords={DECLARE, LOOP, TYPE, FOR, IF, IS, OPEN, FETCH, DBMS_OUTPUT, PUT_LINE}]
SQL> select * from temp;

  NUM_COL1   NUM_COL2 CHAR_COL
---------- ---------- -------------------------------------------------------
		 5000 KING
\end{lstlisting}

\newpage
\section{TP3}
\subsection{Exercice 1 (A)}
Code :
\begin{lstlisting}[language=SQL,
    deletekeywords={char},
    morekeywords={DECLARE, LOOP, TYPE, FOR, IF, IS, OPEN, FETCH, DBMS_OUTPUT, PUT_LINE}]
CREATE OR REPLACE PROCEDURE createdept_zangla(num IN NUMBER, name IN VARCHAR2, loc IN VARCHAR2)
IS
    d NUMBER;
BEGIN
    SELECT deptno INTO d FROM dept WHERE deptno = num;
    RAISE_APPLICATION_ERROR(-20001, 'Numero de departement deja existant');
    EXCEPTION 
        WHEN NO_DATA_FOUND THEN
            INSERT INTO dept VALUES(num, name, loc);
END;
/
\end{lstlisting}

Résultat :
\begin{lstlisting}[language=SQL,
    morekeywords={DECLARE, LOOP, TYPE, FOR, IF, IS, OPEN, FETCH, DBMS_OUTPUT, PUT_LINE}]
## Déjà existant ##
SQL> exec createdept_zangla(30, 'SALES', 'CHICAGO');
BEGIN createdept_zangla(30, 'SALES', 'CHICAGO'); END;

*
ERREUR a la ligne 1 :
ORA-20001: Numero de departement deja existant
ORA-06512: a "BD10.CREATEDEPT_ZANGLA", ligne 6
ORA-06512: a ligne 1


## Ajout ##
SQL> exec createdept_zangla(16, 'VACHE', 'GUERET');

Procedure PL/SQL terminee avec succes.

SQL> select * from dept;

DEPTNO 	    DNAME	      LOC
---------- -------------- -------------
    10 		ACCOUNTING	  NEW YORK
    20 		RESEARCH	  DALLAS
    30 		SALES	      CHICAGO
    16 		VACHE	      GUERET    
\end{lstlisting}

\subsection{Exercice 2 (A)}
Code :
\begin{lstlisting}[language=SQL,
    deletekeywords={char},
    morekeywords={DECLARE, LOOP, TYPE, FOR, IF, IS, OPEN, FETCH, DBMS_OUTPUT, PUT_LINE}]
CREATE OR REPLACE FUNCTION salok_zangla(jobselect in VARCHAR2, salaire in NUMBER) RETURN NUMBER 
IS
    mi NUMBER;
    ma NUMBER;
BEGIN
    SELECT lsal, hsal INTO mi, ma FROM salintervalle_f2 WHERE job = jobselect;
    IF salaire >= mi AND salaire <= ma THEN RETURN 1; 
    ELSE RETURN 0;
    END IF;
    EXCEPTION WHEN NO_DATA_FOUND THEN RETURN 0;
END;
/
\end{lstlisting}

Résultat :
\begin{lstlisting}[language=SQL,
    morekeywords={DECLARE, LOOP, TYPE, FOR, IF, IS, OPEN, FETCH, DBMS_OUTPUT, PUT_LINE}]
SQL> select * from salintervalle_f2;

JOB		LSAL	   HSAL
--------- ---------- ----------
ANALYST 	2500	   3000
CLERK		 900	   1300
MANAGER 	2400	   3000
PRESIDENT	4500	   4900
SALESMAN	1200	   1700

SQL> variable vrai number;
SQL> execute :vrai := salok_zangla('ANALYST', 2900);

Procedure PL/SQL terminee avec succes.

SQL> print vrai;

      VRAI
----------
	 1
SQL> variable faux number;
SQL> execute :faux := salok_zangla('PRESIDENT', 4000);

Procedure PL/SQL terminee avec succes.

SQL> print faux;

      FAUX
----------
	 0
\end{lstlisting}

\subsection{Exercice 3 (A)}
Code :
\begin{lstlisting}[language=SQL,
    deletekeywords={char},
    morekeywords={DECLARE, LOOP, TYPE, FOR, IF, IS, OPEN, FETCH, DBMS_OUTPUT, PUT_LINE}]
CREATE OR REPLACE PROCEDURE raisesalary_zangla(emp_id IN NUMBER, amount IN NUMBER)
IS
    e NUMBER;
    j VARCHAR(9);
    s NUMBER;
    possible NUMBER;
BEGIN
    SELECT empno, job, sal INTO e, j, s FROM emp WHERE empno = emp_id;
    possible := salok_zangla(j, s + amount);
    IF possible = 1 THEN 
        UPDATE emp SET sal = s + amount WHERE empno = e;
    ELSE
        RAISE_APPLICATION_ERROR(-20002, 'Maximum deja atteint');
    END IF;
END;
/
\end{lstlisting}

Résultat :
\begin{lstlisting}[language=SQL,
    morekeywords={DECLARE, LOOP, TYPE, FOR, IF, IS, OPEN, FETCH, DBMS_OUTPUT, PUT_LINE}]
SQL> select * from emp where empno=7876;

    EMPNO ENAME      JOB	       MGR HIREDATE	   SAL	     COMM     DEPTNO
---------- ---------- --------- ---------- -------- ---------- ---------- ----------
     7876 ADAMS      CLERK	      7788 13/07/87	  1100			           20

SQL> execute raisesalary_zangla(7876, 100);

SQL> select * from emp where empno=7876;

    EMPNO ENAME      JOB	       MGR HIREDATE	   SAL	     COMM     DEPTNO
---------- ---------- --------- ---------- -------- ---------- ---------- ----------
     7876 ADAMS      CLERK	      7788 13/07/87	  1200			            20

SQL> execute raisesalary_zangla(7876, 500);
BEGIN raisesalary_zangla(7876, 500); END;

*
ERREUR a la ligne 1 :
ORA-20002: Maximum deja atteint
ORA-06512: a "BD10.RAISESALARY_ZANGLA", ligne 13
ORA-06512: a ligne 1
\end{lstlisting}

\subsection{Exercice 4 (B)}
Code :
\begin{lstlisting}[language=SQL,
    deletekeywords={char},
    morekeywords={DECLARE, LOOP, TYPE, FOR, IF, IS, OPEN, FETCH, DBMS_OUTPUT, PUT_LINE}]
DECLARE
	Cursor c IS SELECT table_name FROM user_tables WHERE table_name NOT LIKE '%_OLD';
	Cursor cold IS SELECT table_name FROM user_tables WHERE table_name LIKE '%_OLD';
	t_name user_tables.table_name%TYPE;
BEGIN
	OPEN cold;
	LOOP
		FETCH cold INTO t_name;
		EXIT WHEN (cold%NOTFOUND);
		EXECUTE IMMEDIATE 'DROP TABLE ' || t_name;
	END LOOP;
	OPEN c;
	LOOP
		FETCH c INTO t_name;
		EXIT WHEN(c%NOTFOUND); 
		EXECUTE IMMEDIATE 'CREATE TABLE ' || t_name || '_old AS (SELECT * FROM ' || t_name || ')';
		DBMS_OUTPUT.PUT_LINE(t_name);
	END LOOP;
END;
/
\end{lstlisting}

Résultat (avec plusieurs exécutions pour vérifier) :
\begin{lstlisting}[language=SQL,
    morekeywords={DECLARE, LOOP, TYPE, FOR, IF, IS, OPEN, FETCH, DBMS_OUTPUT, PUT_LINE}]
SQL> select table_name from user_tables;

TABLE_NAME
--------------------------------------------------------------------------------
AUTEURS
EXCEPTIONS
TEMP
SALINTERVALLE_F2
OUVRAGE
AUTEUR_OUVRAGE
DEPT
EMP

SQL> start b.sql

Procedure PL/SQL terminee avec succes.

SQL> select table_name from user_tables;

TABLE_NAME
--------------------------------------------------------------------------------
AUTEURS
EXCEPTIONS
TEMP
SALINTERVALLE_F2
OUVRAGE
AUTEUR_OUVRAGE
DEPT
EMP
AUTEURS_OLD
AUTEUR_OUVRAGE_OLD
DEPT_OLD
EMP_OLD
OUVRAGE_OLD
SALINTERVALLE_F2_OLD
TEMP_OLD
EXCEPTIONS_OLD

SQL> start b.sql

Procedure PL/SQL terminee avec succes.

SQL> select table_name from user_tables;

TABLE_NAME
--------------------------------------------------------------------------------
AUTEURS
EMP_OLD
EXCEPTIONS
AUTEUR_OUVRAGE_OLD
TEMP
SALINTERVALLE_F2
OUVRAGE
AUTEUR_OUVRAGE
DEPT
EMP
TEMP_OLD
DEPT_OLD
SALINTERVALLE_F2_OLD
AUTEURS_OLD
OUVRAGE_OLD
EXCEPTIONS_OLD
\end{lstlisting}

\newpage
\section{TP4}
\subsection{Exercice Package}
Code :
\begin{lstlisting}[language=SQL,
    deletekeywords={char},
    morekeywords={DECLARE, LOOP, TYPE, FOR, IF, IS, OPEN, FETCH, DBMS_OUTPUT, PUT_LINE}]
CREATE OR REPLACE PACKAGE zangla AS
	TYPE emp_cursor IS RECORD (emp_id NUMBER, nom VARCHAR2(10));
	CURSOR emp_par_dep_zangla(dep IN NUMBER) RETURN emp_cursor;
	PROCEDURE raise_salary_zangla(emp_id IN NUMBER, amount IN NUMBER);
	PROCEDURE afficher_emp_zangla(deptno IN NUMBER);
END;
/

CREATE OR REPLACE PACKAGE BODY zangla AS
	CURSOR emp_par_dep_zangla(dep IN NUMBER) RETURN emp_cursor IS 
		SELECT empno, ename FROM emp WHERE deptno = dep;
	PROCEDURE raise_salary_zangla(emp_id IN NUMBER, amount IN NUMBER) IS
		e NUMBER;
		j VARCHAR(9);
		s NUMBER;
		possible NUMBER;
	BEGIN
		SELECT empno, job, sal INTO e, j, s FROM emp WHERE empno = emp_id;
		possible := salok_zangla(j, s + amount);
		IF possible = 1 THEN 
			UPDATE emp SET sal = s + amount WHERE empno = e;
		ELSE
			RAISE_APPLICATION_ERROR(-20002, 'Maximum deja atteint');
		END IF;
	END;
	PROCEDURE afficher_emp_zangla(deptno IN NUMBER) IS
		emp_id NUMBER;
		nom VARCHAR2(9);
	BEGIN
		OPEN emp_par_dep_zangla(deptno);
		LOOP
			FETCH emp_par_dep_zangla INTO emp_id, nom;
			EXIT WHEN(emp_par_dep_zangla%NOTFOUND);
			DBMS_OUTPUT.PUT_LINE(emp_id || ' : ' || nom);
		END LOOP;
		CLOSE emp_par_dep_zangla;
	END;
END;
/


\end{lstlisting}

Résultat procedure raise\_salary :
\begin{lstlisting}[language=SQL,
    morekeywords={DECLARE, LOOP, TYPE, FOR, IF, IS, OPEN, FETCH, DBMS_OUTPUT, PUT_LINE}]
Depuis la table emp:
    EMPNO ENAME      JOB	       MGR HIREDATE	   SAL	     COMM     DEPTNO
---------- ---------- --------- ---------- -------- ---------- ---------- ----------
    7900 JAMES      CLERK	      7698 03/12/81	  1050			           30

Depuis la table salintervalle_f2:
JOB		LSAL	   HSAL
--------- ---------- ----------
CLERK		 900	   1300

SQL> exec zangla.raise_salary_zangla(7900, 50)

Procedure PL/SQL terminee avec succes.

Depuis la table emp:
EMPNO ENAME      JOB	       MGR HIREDATE	   SAL	     COMM     DEPTNO
---------- ---------- --------- ---------- -------- ---------- ---------- ----------
7900 JAMES      CLERK	      7698 03/12/81	  1100			          30
\end{lstlisting}

\newpage
Résultat procedure afficher\_emp :
\begin{lstlisting}[language=SQL,
    morekeywords={DECLARE, LOOP, TYPE, FOR, IF, IS, OPEN, FETCH, DBMS_OUTPUT, PUT_LINE}]
SQL> select * from emp;

    EMPNO ENAME      JOB	       MGR HIREDATE	   SAL	     COMM     DEPTNO
---------- ---------- --------- ---------- -------- ---------- ---------- ----------
     7369 SMITH      CLERK	      7902 17/12/80	   800			            20
     7499 ALLEN      SALESMAN	  7698 20/02/81	  1600	      300	        30
     7521 WARD       SALESMAN	  7698 22/02/81	  1250	      500	        30
     7566 JONES      MANAGER	  7839 02/04/81	  2975			            20
     7654 MARTIN     SALESMAN	  7698 28/09/81	  1250	     1400	        30
     7698 BLAKE      MANAGER	  7839 01/05/81	  2850			            30
     7782 CLARK      MANAGER	  7839 09/06/81	  2450			            10
     7788 SCOTT      ANALYST	  7566 13/07/87	  3000			            20
     7839 KING       PRESIDENT 	       17/11/81	  5000			            10
     7844 TURNER     SALESMAN	  7698 08/09/81	  1500		    0           30
     7876 ADAMS      CLERK	      7788 13/07/87	  1200			            20
     7900 JAMES      CLERK	      7698 03/12/81	  1100			            30
     7902 FORD       ANALYST	  7566 03/12/81	  3000			            20
     7934 MILLER     CLERK	      7782 23/01/82	  1300			            10
     7000 Zangla     SALESMAN	  7566 17/12/80	  2200			            20

SQL> exec zangla.afficher_emp_zangla(20)
    7369 : SMITH
    7566 : JONES
    7788 : SCOTT
    7876 : ADAMS
    7902 : FORD
    7000 : Zangla
    
    Procedure PL/SQL terminee avec succes.
\end{lstlisting}


\subsection{Exercice Trigger 1}
Code :
\begin{lstlisting}[language=SQL,
    deletekeywords={char},
    morekeywords={DECLARE, LOOP, TYPE, FOR, IF, IS, OPEN, FETCH, DBMS_OUTPUT, PUT_LINE}]
CREATE OR REPLACE TRIGGER raise_zangla
	BEFORE UPDATE ON emp
	FOR EACH ROW
	WHEN (new.sal < old.sal)
	BEGIN
		RAISE_APPLICATION_ERROR(-20003, 'Impossible de diminuer le salaire !');
END;
/
\end{lstlisting}

Résultat :
\begin{lstlisting}[language=SQL,
    morekeywords={DECLARE, LOOP, TYPE, FOR, IF, IS, OPEN, FETCH, DBMS_OUTPUT, PUT_LINE}]
SQL> update emp set sal=1100 where empno=7934;
update emp set sal=1100 where empno=7934
        *
ERREUR a la ligne 1 :
ORA-20003: Impossible de diminuer le salaire !
ORA-06512: a "BD10.RAISE_ZANGLA", ligne 2
ORA-04088: erreur lors d'execution du declencheur 'BD10.RAISE_ZANGLA'
\end{lstlisting}

\end{document}

