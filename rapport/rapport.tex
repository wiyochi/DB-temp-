\documentclass{article}

\usepackage[utf8x]{inputenc}
\usepackage[T1]{fontenc}
\usepackage[francais]{babel}
\usepackage{xcolor}
\usepackage{listings}
\usepackage{mathptmx}
\usepackage{anyfontsize}
\usepackage{t1enc}
\usepackage[top=2cm, bottom=2cm, left=2cm, right=2cm]{geometry}
\usepackage{titlesec}
\usepackage{titling}
\usepackage{graphicx}

\newcommand{\changeurlcolor}[1]{\hypersetup{urlcolor=#1}}

\renewcommand\maketitlehooka{\null\mbox{}\vfill}
\renewcommand\maketitlehookd{\vfill\null}

\definecolor{codegreen}{rgb}{0,0.6,0}
\definecolor{codegray}{rgb}{0.5,0.5,0.5}
\definecolor{codepurple}{rgb}{0.58,0,0.82}
\definecolor{backcolour}{rgb}{0.95,0.95,0.92}
\definecolor{codekeywords}{rgb}{0.1,0.53,0.92}

\lstdefinestyle{c++}{
    backgroundcolor=\color{backcolour},   
    commentstyle=\color{codegreen},
    keywordstyle=\color{codekeywords},
    numberstyle=\tiny\color{codegray},
    stringstyle=\color{codepurple},
    basicstyle=\ttfamily\footnotesize,
    breakatwhitespace=false,         
    breaklines=true,                 
    captionpos=b,                    
    keepspaces=true,                 
    numbers=left,                    
    numbersep=5pt,                  
    showspaces=false,                
    showstringspaces=false,
    showtabs=false,                  
    tabsize=2,
    texcl=false,
    inputencoding=utf8,
    extendedchars=true,
    literate=
  {á}{{\'a}}1 {é}{{\'e}}1 {í}{{\'i}}1 {ó}{{\'o}}1 {ú}{{\'u}}1
  {Á}{{\'A}}1 {É}{{\'E}}1 {Í}{{\'I}}1 {Ó}{{\'O}}1 {Ú}{{\'U}}1
  {à}{{\`a}}1 {è}{{\`e}}1 {ì}{{\`i}}1 {ò}{{\`o}}1 {ù}{{\`u}}1
  {À}{{\`A}}1 {È}{{\'E}}1 {Ì}{{\`I}}1 {Ò}{{\`O}}1 {Ù}{{\`U}}1
  {ä}{{\"a}}1 {ë}{{\"e}}1 {ï}{{\"i}}1 {ö}{{\"o}}1 {ü}{{\"u}}1
  {Ä}{{\"A}}1 {Ë}{{\"E}}1 {Ï}{{\"I}}1 {Ö}{{\"O}}1 {Ü}{{\"U}}1
  {â}{{\^a}}1 {ê}{{\^e}}1 {î}{{\^i}}1 {ô}{{\^o}}1 {û}{{\^u}}1
  {Â}{{\^A}}1 {Ê}{{\^E}}1 {Î}{{\^I}}1 {Ô}{{\^O}}1 {Û}{{\^U}}1
  {œ}{{\oe}}1 {Œ}{{\OE}}1 {æ}{{\ae}}1 {Æ}{{\AE}}1 {ß}{{\ss}}1
  {ç}{{\c c}}1 {Ç}{{\c C}}1 {ø}{{\o}}1 {å}{{\r a}}1 {Å}{{\r A}}1
  {€}{{\EUR}}1 {£}{{\pounds}}1,
}
\lstset{style=c++}


\title{Développement des Bases de Données}
\author{Arquillière Mathieu}

\begin{document}

\begin{titlepage}
  \maketitle
\end{titlepage}

\newpage

\section{TP1}

\subsection{Exercice 1}
Code :
\begin{lstlisting}[language=SQL,
    morekeywords={DECLARE, LOOP, TYPE, FOR, IF, IS, OPEN, FETCH, DBMS_OUTPUT, PUT_LINE}]
DECLARE
    nom emp.ename%TYPE;
    salaire emp.sal%TYPE;
    commission emp.comm%TYPE;
    departement dept.dname%TYPE;
BEGIN 
    SELECT ename,sal,comm,dname INTO nom,salaire,commission,departement FROM Emp NATURAL JOIN Dept
    WHERE ename='MILLER';
    DBMS_OUTPUT.PUT_LINE('Nom : ' || nom || ' Salaire : ' || salaire || ' Commission : ' || commission || 'Departement : ' || departement);
END;
/
\end{lstlisting}

Résultat :
\begin{lstlisting}[language=SQL,
    morekeywords={DECLARE, LOOP, TYPE, FOR, IF, IS, OPEN, FETCH, DBMS_OUTPUT, PUT_LINE}]
\end{lstlisting}

\subsection{Exercice 2}
Code :
\begin{lstlisting}[language=SQL,
    deletekeywords={char},
    morekeywords={DECLARE, LOOP, TYPE, FOR, IF, IS, OPEN, FETCH, DBMS_OUTPUT, PUT_LINE}]
DECLARE
    num1 temp.num_col1%TYPE;
    num2 temp.num_col2%TYPE;
    char temp.char_col%TYPE;
BEGIN
	FOR i IN 1..10 LOOP
		IF MOD(i, 2) = 0 THEN
			INSERT INTO temp VALUES (i, i * 100, CONCAT(TO_CHAR(i), ' est pair'));
		ELSE
			INSERT INTO temp VALUES (i, i * 100, CONCAT(TO_CHAR(i), ' est impair'));
		END IF;
	END LOOP;
	COMMIT;
END;
/
\end{lstlisting}

Résultat :
\begin{lstlisting}[language=SQL,
    morekeywords={DECLARE, LOOP, TYPE, FOR, IF, IS, OPEN, FETCH, DBMS_OUTPUT, PUT_LINE}]
\end{lstlisting}

\subsection{Exercice 3}
Code :
\begin{lstlisting}[language=SQL,
    deletekeywords={char},
    morekeywords={DECLARE, LOOP, TYPE, FOR, IF, IS, OPEN, FETCH, DBMS_OUTPUT, PUT_LINE}]
DECLARE
	Cursor c IS SELECT sal, empno, ename FROM emp ORDER BY sal DESC;
	salaire emp.sal%TYPE;
	numero emp.empno%TYPE;
	nom emp.ename%TYPE;
BEGIN
	OPEN c;
	FOR i IN 1..5 LOOP
		FETCH c INTO salaire, numero, nom;
		INSERT INTO temp VALUES (salaire, numero, nom);
	END LOOP;
END;
/
\end{lstlisting}

Résultat :
\begin{lstlisting}[language=SQL,
    morekeywords={DECLARE, LOOP, TYPE, FOR, IF, IS, OPEN, FETCH, DBMS_OUTPUT, PUT_LINE}]
\end{lstlisting}

\subsection{Exercice 4}
Code :
\begin{lstlisting}[language=SQL,
    deletekeywords={char},
    morekeywords={DECLARE, LOOP, TYPE, FOR, IF, IS, OPEN, FETCH, DBMS_OUTPUT, PUT_LINE}]
DECLARE
    Cursor c IS SELECT UNIQUE sal, NVL(comm, 0), empno, ename FROM emp WHERE sal + NVL(comm, 0) > 2000;
    salaire emp.sal%TYPE;
    numero emp.empno%TYPE;
    nom emp.ename%TYPE;
    comm emp.comm%TYPE;
BEGIN
	OPEN c;
	LOOP
		FETCH c INTO salaire, comm, numero, nom;
		INSERT INTO temp VALUES (salaire + comm, numero, nom);
		EXIT WHEN (c%notfound);
	END LOOP;
END;
/
\end{lstlisting}

Résultat :
\begin{lstlisting}[language=SQL,
    morekeywords={DECLARE, LOOP, TYPE, FOR, IF, IS, OPEN, FETCH, DBMS_OUTPUT, PUT_LINE}]
\end{lstlisting}

\subsection{Exercice 5}
Code :
\begin{lstlisting}[language=SQL,
    deletekeywords={char},
    morekeywords={DECLARE, LOOP, TYPE, FOR, IF, IS, OPEN, FETCH, DBMS_OUTPUT, PUT_LINE}]
DECLARE
	Cursor c IS SELECT sal, ename, empno, mgr FROM emp;
	salaire emp.sal%TYPE;
	nom emp.ename%TYPE;
	empno emp.empno%TYPE;
	mgrEmp emp.mgr%TYPE;
	chaineMgr emp.mgr%TYPE;
BEGIN
	OPEN c;
	LOOP
		FETCH c INTO salaire, nom, empno, mgrEmp;
		EXIT WHEN(c%NOTFOUND);
		IF salaire >= 4000 THEN
			DBMS_OUTPUT.PUT_LINE('Salaire: ' || salaire || ' Nom: ' || nom || ' no: ' || empno || ' mgr: ' || mgrEmp);
			SELECT mgr INTO chaineMgr FROM emp WHERE empno=7902;
			LOOP
				EXIT WHEN(chaineMgr IS NULL OR chaineMgr=mgrEmp);
				SELECT mgr INTO chaineMgr FROM emp where empno=chaineMgr;
			END LOOP;
			IF chaineMgr=mgrEmp OR mgrEmp IS NULL THEN
				INSERT INTO temp VALUES (null, salaire, nom);
			END IF;
		END IF;
	END LOOP;
END;
/
\end{lstlisting}

Résultat :
\begin{lstlisting}[language=SQL,
    morekeywords={DECLARE, LOOP, TYPE, FOR, IF, IS, OPEN, FETCH, DBMS_OUTPUT, PUT_LINE}]
\end{lstlisting}

\newpage
\section{TP2}
\subsection{Exercice 1}
Code :
\begin{lstlisting}[language=SQL,
    deletekeywords={char},
    morekeywords={DECLARE, LOOP, TYPE, FOR, IF, IS, OPEN, FETCH, DBMS_OUTPUT, PUT_LINE}]
CREATE OR REPLACE PROCEDURE createdept_zangla(num IN NUMBER, name IN VARCHAR2, loc IN VARCHAR2)
IS
    d NUMBER;
BEGIN
    SELECT deptno INTO d FROM dept WHERE deptno = num;
    RAISE_APPLICATION_ERROR(-20001, 'Numero de departement deja existant');
    EXCEPTION 
        WHEN NO_DATA_FOUND THEN
            INSERT INTO dept VALUES(num, name, loc);
END;
/

/*exec createdept_zangla(12, 'TEst', 'Aubiere');*/

CREATE OR REPLACE FUNCTION salok_zangla(jobselect in VARCHAR2, salaire in NUMBER) RETURN NUMBER 
IS
    j VARCHAR2(9);
    mi NUMBER;
    ma NUMBER;
BEGIN
    SELECT job, lsal, hsal INTO j, mi, ma FROM salintervalle_f2 WHERE mi <= salaire AND ma >= salaire;
    RETURN 1; 
    EXCEPTION WHEN NO_DATA_FOUND THEN RETURN 0;
END;
/

select salok_zangla(job, 2800) FROM salintervalle_f2;
\end{lstlisting}

Résultat :
\begin{lstlisting}[language=SQL,
    morekeywords={DECLARE, LOOP, TYPE, FOR, IF, IS, OPEN, FETCH, DBMS_OUTPUT, PUT_LINE}]
\end{lstlisting}
\end{document}